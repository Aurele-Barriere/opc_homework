\documentclass{article}
\usepackage{fullpage}
\usepackage{listings}
\usepackage[dvipsnames]{xcolor}
\usepackage{graphicx}
\usepackage[utf8]{inputenc}
\usepackage[T1]{fontenc}

\lstset{language=C,
                basicstyle=\ttfamily\footnotesize,
                keywordstyle=\color{RedViolet}\ttfamily,
                stringstyle=\color{Violet}\ttfamily,
                commentstyle=\color{MidnightBlue}\ttfamily,
                morecomment=[l][\color{magenta}]{\#},
                morekeywords={bool, list, int, and, print},
                emph={spontaneous_choice, take_snapshot, process_event%  
                },emphstyle={\color{Maroon}}
}


\title{DM Module OPC}
\author{Aur\`ele Barri\`ere}
\date{1er février 2019}


\begin{document}
\maketitle
\vspace{1cm}

\section*{Question 1}
Les commandes correspondent aux opérations polyhédriques suivantes:
\begin{lstlisting}
[N] -> { [i, j] : 0 < i < N and j > i and 15 <= j <= 33 and j < N }
[N] -> { 437 : N >= 35; (-90 - 3/2 * N + 1/2 * N^2) : 16 <= N <= 34 }
[N] -> { [ip, jp] : 16*floor((jp)/16) = jp and ip >= 0 and 16ip <= -2 + N and 0 <= jp <= 15 }
\end{lstlisting}

Le premier domaine affiché est l'intersection des deux domaines \lstinline{A} et \lstinline{B}.
On y retrouve bien les contraintes des deux domaines. La contrainte existentielle a été supprimée: elle ne contraint pas le domaine A, puisque pour chaques valeurs $(i,j)\in A$, $i+j>6$. Les contraintes redondantes (comme $i\geq 0$ alors qu'on a aussi $i>0$) ont été supprimées.

Le second est le cardinal du précédent domaine. Comme attendu, il dépend de la valeur de \lstinline{N}.

Enfin, \lstinline{E} est une transformation de domaine de vecteurs d'itération.
Appliqué à \lstinline{B}, on obtient l'ensemble des $(i',j')$ tels qu'il existe $(i,j)\in B$ et $i'=\frac{i}{16}$ et $j'=i\% 16$.

\section*{Question 2}

Étant donnés des instructions \lstinline{S} et \lstinline{T} dont chaque domaine d'exécution est décrit dans \lstinline{Domain}, le code donné va générer le code des boucles qui effectuent ces instructions dans l'ordre donné par \lstinline{Schedule}. \lstinline{S} et \lstinline{T} sont paramétrés par une ou deux valeurs, $i,j$ ou $k$, des variables d'itération.

Dans ce cas, l'ordonnancement précise bien que les instructions de \lstinline{S} doivent êtres faites d'abord (0 en première coordonnée), et par ordre de \lstinline{k}.
Ainsi, le code généré commence par une boucle qui parcoure le domaine de \lstinline{S} par ordre de \lstinline{k}.

Ensuite, l'ordonnancement précise que les itérations de \lstinline{T} doivent être faites par ordre croissant lexicographique de $(j,i+j)$.
Le code généré respecte cet ordre.
En effet, pour respecter cet ordre, il faut fixer la seconde variable et faire augmenter la première jusqu'à $N$, puis augmenter la seconde variable et recommencer.
C'est ainsi que procède le code généré: \lstinline{-c1 + c2} commence à 0 et est incrémenté dans la boucle interne, alors que \lstinline{c1} (qui correspond à $j$) est augmenté dans la boucle externe.

Code généré:
\begin{lstlisting}
  for (int c1 = 0; c1 < 2 * n - 1; c1 += 1)
    S(c1);
  for (int c1 = 0; c1 < n; c1 += 1)
    for (int c2 = c1; c2 < n + c1; c2 += 1)
      T(-c1 + c2, c1);
\end{lstlisting}

\section*{Question 3}

On écrit les commandes suivantes:
\begin{lstlisting}
D := [N] -> { [i,j,k] : 0<=i<j and 0<=j<N and i<k<j};
P := { [i,j,k] -> [ip,kp] : ip=i and kp=k};
print P(D);
C:=card(P(D));
print C;
\end{lstlisting}

qui donnent le résultat suivant:
\begin{lstlisting}
[N] -> { [ip, kp] : ip >= 0 and ip < kp <= -2 + N }
[N] -> { (1 - 3/2 * N + 1/2 * N^2) : N >= 3 }
\end{lstlisting}

\lstinline{P} représente la projection.
D'un domaine à 3 dimensions, on n'en conserve que deux.
Ainsi, on retient tous les $(i,k)$ tels que $\exists j, (i,j,k)\in D$.
Enfin la fonction \lstinline{card} permet de compter les points du domaine obtenu.

La projection contient donc $1-\frac{3\times N}{2} + \frac{N^2}{2}$ points, lorsque $N>3$. Sinon, la proprojection est l'ensemble vide.


\section*{Question 4}

\begin{lstlisting}
Domain := [N] -> {
    S0[i] : 0<=i<N;
    S1[i, j] : 0<=i<N and 1<=j<i;
};

Schedule := [N] -> {
   S0[i] -> [i,0];
   S1[i,j] -> [i, j];
};

Read := [N] -> {
  S1[i,j] -> Y[j];
  S1[i,j] -> T[i-j];
} * Domain;

Write := [N] -> {
  S0[i] -> Y[i];
  S1[i,j] -> Y[i];
} * Domain;

Before := Schedule << Schedule;

RaW := (Write . (Read^-1)) * Before;
WaW := (Write . (Write^-1)) * Before;
WaR := (Read . (Write^-1)) * Before;

print RaW;
print WaW;
print WaR;
\end{lstlisting}

\lstinline{Domain} définit les domaines d'itérations des instructions \lstinline{S0} et \lstinline{S1}.
Ces contraintes sont tirées directement du flot de contrôle du programme: on regarde les guardes des boucles.

\lstinline{Schedule} décrit l'ordonnancement original (sans transformation de boucle).
Dans ce programme, pour chaque valeur de $i$ on éxécute d'abord \lstinline{S0}, puis toutes les éxécutions de \lstinline{S1} pour $i$ et chaque valeur de $j$ par ordre croissant.
Cela correspond à l'ordre lexicographique donné ici. Comme $j>0$, on est garanti d'exécuter \lstinline{S0} avant les exécutions de \lstinline{S1} correspondantes.

Les relations \lstinline{Read} et \lstinline{Write} définissent les lectures et écritures en mémoires.
Ces relations sont tirées directement de la syntaxe du programme.
Par exemple, lors de l'itération $(i,j)$ de \lstinline{S1}, on écrit \lstinline{Y[i]}.
La relation \lstinline{Write} contient alors \lstinline{ S1[i,j] -> Y[i];}.
Cette définition est intersectée avec \lstinline{Domain} pour n'inclure que les itérations qui sont en effet exécutées. Par exemple, \lstinline{Y[N+1]} ne sera pas écrit.

\lstinline{Before} exprime qu'une instruction est lancée avant une autre dans l'ordonnancement.
Il s'agit du domaine des paires $(v,v')$ de vecteurs d'itérations tels que $v$ est exécuté avant $v'$.
Enfin, les définitions de \lstinline{RaW}, \lstinline{WaW} et \lstinline{WaR} représentent les paires de vecteurs d'itérations où, par exemple, une case est lue alors qu'elle avait été écrite avant.

Résultats:
\begin{lstlisting}
[N] -> { S0[i] -> S1[ip, i] : 0 < i < N and ip > i and 0 <= ip < N;
         S1[i, j] -> S1[ip, i] : 0 < i < N and 0 < j < i and ip > i and 0 <= ip < N }
[N] -> { S1[i, j] -> S1[i, jp] : 0 <= i < N and 0 < j < i and jp > j and 0 < jp < i;
         S0[i] -> S1[i, j] : 0 <= i < N and 0 < j < i }
[N] -> {  }
\end{lstlisting}

On observe en particulier aucune dépendance \lstinline{WAR}: dans cet ordonnancement, on ne lit une case \lstinline{Y[j]} qu'après l'avoir écrite (car $j<i$ pour toutes itérations).

\section*{Question 5}
\begin{lstlisting}
TransfoSchedule := [N] -> {
  S0[i] -> [i,0,0];
  S1[i,j] -> [j,1,i+j];
};

codegen(TransfoSchedule * Domain);
\end{lstlisting}

On décrit le nouvel ordonnancement. Le code généré est le suivant:

\begin{lstlisting}
for (int c0 = 0; c0 < N; c0 += 1) {
  S0(c0);
  if (c0 >= 1)
    for (int c2 = 2 * c0 + 1; c2 < N + c0; c2 += 1)
      S1(-c0 + c2, c0);
}
\end{lstlisting}

La condition \lstinline{if} apparaît car seul \lstinline{S0} doit être exécutée lors de la première itération: il n'existe pas de $j\leq 0$ dans le domaine.

\section*{Question 6}
\begin{lstlisting}
TransfoSchedule := [N] -> {
  S0[i] -> [i,0,0];
  S1[i,j] -> [j,1,i+j];
};

TransfoBefore := TransfoSchedule << TransfoSchedule;

print "RaW?";
print RaW <= TransfoBefore;

print "WaW?";
print WaW <= TransfoBefore;

print "WaR?";
print WaR <= TransfoBefore;
\end{lstlisting}

Le nouvel ordonnancement est comparé aux définitions de dépendances calculées avec l'ordonnancement original (question 4).
En particulier, on veut vérifier que, pour chaque paire $(v,v')$ de vecteurs d'itérations dans une dépendance, cette paire est aussi dans \lstinline{TransfoBefore}, ce qui veut dire que $v$ est bien exécuté avant $v'$ dans le nouvel ordonnancement. On utilise donc simplement l'inclusion \lstinline{<=} entre ces ensembles de vecteurs.

Résultat:
\begin{lstlisting}
"RaW?"
True
"WaW?"
False
"WaR?"
True
\end{lstlisting}

On remarque qu'une dépendance \lstinline{WaW} n'est pas satisfaite.
En effet, avec ce nouvel ordonnancement on peut exécuter $S1(1,0)$ avant $S0(1)$, et donc écrire dans \lstinline{Y[1]} dans l'ordre inverse du programme original.

Ainsi, la sémantique du programme n'est pas conservé par cette transformation de boucles.

\section*{Question 7}

\begin{lstlisting}
DS0 := [N] -> { S0[i] : 0<=i<N; };
DS1 := [N] -> { S1[i, j] : 0<=i<N and 1<=j<i; };
print (card (DS0));
print (card (DS1));
\end{lstlisting}

On commence par définir les domaines d'itérations de chaque instruction. Ces définitions proviennent de la syntaxe du programme, et dépendent de N.
Pour chaque élément dans ces domaines, l'instruction correspondante est exécutée exactement une fois.
Ainsi, on peut donc se contenter de calculer le cardinal de ces deux ensembles.

Résultat:
\begin{lstlisting}
[N] -> { N : N > 0 }
[N] -> { (1 - 3/2 * N + 1/2 * N^2) : N >= 3 }
\end{lstlisting}

\lstinline{S0} est donc exécutée $N$ fois quand $N>0$, et \lstinline{S1} $1-\frac{3\times N}{2} + \frac{N^2}{2}$ fois si $N\geq 3$.

\section*{Question 8}
\begin{lstlisting}
ArrayAccess := [N] -> {
  S1[i,j] -> T[i-j];
} * DS1;

print(ran(ArrayAccess));
print(card(ran(ArrayAccess)));
\end{lstlisting}

La relation \lstinline{ArrayAccess} associe à chaque itération de \lstinline{S1} la case écrite de \lstinline{T} associée.
On intersecte cette définition avec \lstinline{DS1} (le domaine d'itération de \lstinline{S1}), pour ne conserver que les itérations réellement effectuées par le programme.

Cet ensemble contient tous les vecteurs d'itérations où \lstinline{T} est écrit, mais une case peut être écrite plusieurs fois. On utilise alors la fonctions \lstinline{ran}, qui calcule l'image de cette relation, et donc l'ensemble des cases distinctes de \lstinline{T} écrites.
Enfin, il suffit de compter le cardinal de ce nouvel ensemble.

Résultat:
\begin{lstlisting}
[N] -> { T[i0] : 0 < i0 <= -2 + N }
[N] -> { (-2 + N) : N >= 3 }
\end{lstlisting}

On constate donc que le programme écrit dans $N-2$ cases différentes de \lstinline{T} pour $N\geq 3$.
\lstinline{ran(ArrayAccess)} nous donne en particulier quelles sont ces cases: tous les \lstinline{T[i0]} pour $i_0\in[1,N-2]$.


\section*{Question 9}

\begin{lstlisting}
ArrayAccessIntern := [N,i] -> {
  S1[i,j] -> T[i-j];
} * DS1;

print(card(ran(ArrayAccessIntern)));
\end{lstlisting}

Cette fois-ci, la relation \lstinline{ArrayAccessIntern} décrit les cases de \lstinline{T} accédées dans la boucle interne $i$.
Encore une fois, on utilise \lstinline{ran} pour obtenir l'ensemble de ces cases écrites, et on affiche son cardinal.

Résultat:
\begin{lstlisting}
[N, i] -> { (-1 + i) : 2 <= i < N }
\end{lstlisting}

On constate alors que $i-1$ cases sont écrites pendant l'itération $i$.

\section*{Question 10}

\begin{lstlisting}
ReadT := [p] -> {
  S1[i,j] -> [i,j] : i-j = p;
} * DS1;

print (card(ran(ReadT)));
\end{lstlisting}

\lstinline{ReadT} de $p$ associe à chaque itération de \lstinline{S1} où la case $p$ est accédée le vecteur d'itération correspondant.
On intersecte cette définition avec \lstinline{DS1} pour ne conserver que les itérations réellement effectuées par le programme.

Ensuite, \lstinline{ran(ReadT)} représente exactement l'ensemble des vecteurs d'itérations modifiant la case $p$. Il reste alors à calculer son cardinal.

Résultat:
\begin{lstlisting}
[N, p] -> { ((-1 + N) - p) : 0 < p <= -2 + N }
\end{lstlisting}

Ainsi, il y a $N-1-p$ itérations de \lstinline{S1} qui modifient la case $p$.

\section*{Question 12}
\begin{lstlisting}
currentS0 := [p] -> { S0[p]} * DS0;
nextIt0 := Before * currentS0;
nextDS0 := ran(nextIt0) * DS0;
print (nextDS0);
print (card(nextDS0));

currentS1 := [p,q] -> { S1[p,q]} * DS1;
nextIt1 := Before * currentS1;
nextDS1 := ran(nextIt1) * DS1;
print(nextDS1);
print (card(nextDS1));
\end{lstlisting}

\lstinline{currentS0} associe à $p$ l'itération \lstinline{S0[p]} si elle est dans le domaine de \lstinline{S0}.
\lstinline{nextIt0} associe à cette instruction l'ensemble des itérations du schedule qui sont éxécutées après cette itération. Avec \lstinline{ran}, on récupère exactement les itérations exécutées après \lstinline{S0[p]}.
Cependant, cet ensemble contient aussi les exécutions de \lstinline{S1}. Comme ici on ne s'intéresse qu'au domaine restant pour \lstinline{S0}, on l'intersecte avec \lstinline{DS0}.
Enfin, \lstinline{nextDS0} est donc le domaine des itérations restant à exécuter pour \lstinline{S0}, et on peut calculer son cardinal.
Le même raisonnement nous permet de calculer \lstinline{nextDS1}: le domaine des itérations restant à exécuter pour \lstinline{S1}.

Résultat:
\begin{lstlisting}
[N, p] -> { S0[i] : 0 <= p < N and i > p and 0 <= i < N }
[N, p] -> { ((-1 + N) - p) : 0 <= p <= -2 + N }
[N, p, q] -> { S1[i, j] : 0 <= p < N and 0 < q < p and i > p and 0 <= i < N and 0 < j < i;
               S1[p, j] : 0 <= p < N and 0 < q < p and j > q and 0 < j < p }
[N, p, q] -> { (((-3/2 * N + 1/2 * N^2) + 3/2 * p - 1/2 * p^2) - q) : p <= -2 + N and 0 < q <= -2 + p;
               ((1 - 3/2 * N + 1/2 * N^2) + 1/2 * p - 1/2 * p^2) : q = -1 + p and 2 <= p <= -2 + N;
               ((-1 + p) - q) : p = -1 + N and 0 < q <= -3 + N }
\end{lstlisting}

On constate donc qu'il faut encore exécuter $N-p-1$ fois \lstinline{S0} après avoir exécuté \lstinline{S0[p]}.
Et que le nombre restantes d'exécutions de \lstinline{S1} après avoir exécuté \lstinline{S1[p,q]} dépend de conditions sur $p$ et $q$.



\end{document}
